\documentclass[a4paper,12pt]{article}
\usepackage{xcolor}
\usepackage[colorlinks = true,
            linkcolor = blue,
            urlcolor  = blue,
            citecolor = blue,
            anchorcolor = blue]{hyperref}
\newcommand{\lnk}[3][blue]{\href{#2}{\color{#1}{#3}}}%

\newcommand{\sectionbreak}{\clearpage}

\begin{document}

% Title of the project
\title{Steem Times White Paper}

% Authors
\author{revisesociology,\\ gabriellecd,\\ victorcovrig,\\ warpedpoetic,\\ superoo7,\\ karmachela,\\ livinguktaiwan}
% Created date / last updated date
\date{June 8, 2018}
\maketitle
\sectionbreak
% 1 Rational
\section{Rational}
The steem blockchain is an innovation that stands out in the innovation that is the blockchain technology. Unlike other blockchain related platforms, the steem blockchain is the only one of its kind that offers the average user an opportunity to step through the rabbit hole that is blockchain technology. It enables the noob, the mom, the grandpa, the dude on the street who knows next to nothing about cryptocurrency, blockchain technology or programming, etc, to have access to the innovation, head down the rabbit hole to make money out of it and to build communities out of it. This something that all other active blockchain related projects lack.
\\\\
The steem blockchain flagship project, steemit.com, is a social media unlike any social media out there. It offers users, not only the opportunity to actively market their skills and ideas, but also to interact with a wider base of dedicated and invested users, who are mutually interested in growing and bringing much more energy, beauty, and artistry to the platform and to the lives of the users themselves. It is a community driven project that seeks to bridge the gap between content creators and content consumers, project creators and investors, cultures, art and sciences and between people.
\\\\
To achieve this, a lot has to be done, but for those who have been on the platform for a long time, it is obvious that there are lapses in the process of bridging these gaps. It is obvious that due to the financial benefits inherent in the platform’s core process of interaction, abuse cannot be avoided, and as human beings, it is a sad commentary on our evolution that we know how to take, destroy, kill but do not know how to give, how to maintain, how to manage, how to grow things. 
\\\\
A lot has been said about problems on steemit: abuse of the reward pool, the dismal state of the trending page, the increasing number of inactive accounts, the inactive users with huge SP, the delegation to bidbots and other issues that the platform face but this is not the time or place for a debate. This is the place where we put our foot forward in bridging the gap between content creators and content consumers, between cultures, languages and ideas, between project coordinators and investors. For, we can no longer wait for the whales to come to our aid, if we are going to keep the platform alive. It is our duty to keep this beautiful dream alive and to make it bud, grow and flower into something we will all cherish. 
\\\\
In other to further this goal, we have gathered together a team of individuals from different cultures, different areas of specialization, to form a newspaper for the steem blockchain.
\\\\
We feel and believe that the steem blockchain as a whole is underreported and as a result many of the users do not know or do not have access to the issues on the platform, the consequences of their actions, the beauty to be found on the platform as well as the activities of those elected to stand for us as witnesses. 
\\\\
This will be the first steem newspaper and it intends to focus on the steem blockchain in such a way that the gaps can be bridged and the idea of community that propels the platform is achieved.
\\\\

% 2 members
\section{Members of SteemTimes}
\begin{center}
\begin{tabular} {|p{7em}|p{4em}|p{7em}|p{7.5em}|p{7.5em}|}
\hline
Steemit Name & Country & Skills & Role & Additional Info \\
\hline
@revisesociology & UK & Procrastination/ native English speaker & Updates on ‘what’s occurring on the ‘steem ecosystem’ & Best beard on the blockchain \\
\hline
@gabriellecd & Venezuela & & & \\
\hline
@victorcovrig & Romania & & & \\
\hline
@warpedpoetic & Nigeria & & & \\
\hline
@superoo7 & Australia / Malaysia & Software Development & Creating the whitepaper github repository & High enthusiasm undergraduate student \\
\hline
@karmachela & Indonesia & & & \\
\hline
@lvinguktaiwan & Taiwan & & & \\
\hline
\end{tabular}
\end{center}


\section {Official account}

SteemTimes only have one official account on Steem blockchain which is \textbf{\lnk{https://steemit.com/@steem-times}{@steem-times}}.

\sectionbreak

\section {Aims and Objectives}

The @steem-times, has the following aims and objectives;

\begin{enumerate}
  \item To be the first port of call for all significant developments within the steem ecosystem: from hard forks, DApp developments, whale movements and witness chats, to economic analysis and community projects.
  \item To analyse issues presented as well as discussions that are ongoing on witness chats, discord shows as well as posts by those who have a direct access to the flow of information. 
  \item To showcase projects that are ongoing on the steem blockchain through interviews with project developers, testers who may be engaged in lots of ‘behind the scenes’ work, but not actually be that active on steemit’s front end. With this in mind, we hope to partner with utopian.io as well as project developers all over the steem blockchain in other to gain access to first hand information on the progress of projects.
  \item To identify quality posts and present them for our readers and subscribers to read and follow. This is to create an avenue for content consumers to connect with content creators who have quality content on their blogs.
  \item To identify newbies who show promise and encouraging them along with interviews, guest contributor role for a certain time yet to be ascertained as well as curation. 
  \item To connect with curators from @OCD, @Curie, @sndbox-alpha, etc as well as mentors, editors on the different discord servers, in other to get out a firsthand info on the growth of newbies as well as other issues like curating quality posts, the difficulties in the teaching of newbies as well as curation.
\end{enumerate}

\sectionbreak
\section {Posting and Categories}

There will be 3 weekly issues of the @steem-times, published Monday, Weds and Friday, and each will have three sections:
\begin{enumerate}
    \item (30\% of content, 600 words) Steem Blockchain news which will appear in every issue - this will contain a brief overview of some of the most recent developments on the steem blockchain, such as releases by @steemitblog on major technical developments, new SMT and DApp launches and progress, Witness and Whale movements and comments, and details of new curation and projects launched on steemit itself. 
    \item (20\% of content, 400 words (133 per sub-section)) - Links, with brief summaries, to some of the best posts produced in the the last week under the following categories of steemit. (Might subject to change)
    \begin{enumerate}
        \item Monday: (gabriellecd \& livinguktaiwan)
            \begin{enumerate}
                \item Art and Culture
                \item Writing and Litreature
                \item Sports
            \end{enumerate}
        \item Wednesday (revisesociology \& warpedpoetic)
            \begin{enumerate}
                \item Philosophy and Religion
                \item World Politics
                \item Science and Technology
            \end{enumerate}
        \item Friday (superoo7 \& karmachel)
            \begin{enumerate}
                \item Gossip and Lifestyle
                \item Travel
                \item Gaming
            \end{enumerate}
    \end{enumerate}
    \item (Max 1000 words, 500 words per piece) The third section will be 2 columns, our (the team’s) opinion or views on state of things.

\end{enumerate}

\section {Template}

The template is to be created that a comfortable number of sections are on each post. This is to make it possible to every subscriber to find something to his or her enjoyment on each and every issue we publish. 
\\\\
Sample view of the template at \lnk{https://steemit.com/esteem/@johnsonlai/say-no-more-to-microsoft-word-fe73aa1ca8232}{steemit} and \lnk{https://busy.org/esteem/@johnsonlai/say-no-more-to-microsoft-word-fe73aa1ca8232}{busy.org}. \
The markdown version is available at Official Github repo of SteemTimes: \lnk{https://github.com/superoo7/SteemTimes/blame/master/format.md}{format.md}.

\sectionbreak
\section {Rewards and Payment}


We have 100 SBD as a start up budget for @steem-times and it is proposed that we will use it in the following ways:

\begin{enumerate}
    \item \textbf{Post using the \#steem-times tag}\\Encourage Steemians to post articles about their own opinion and state of things on Steemit and use the tag  \#steem-times.  Since this is a newspaper and not a curation project we may not reward posts using \#steem-times in the long run, but we will do so in the initial few months as a way to promote the newspaper and create awareness.  The reward will be nominal like SBD0.3 to SBD1.  The number of rewarded post each day will vary depending on the volume and quality.  We estimate to spend no more than 25\%  of our budget in this area.
    \item \textbf{Guest columnists}\\If we identify any Steemians from the above program who provides good quality insight and opinions, we will invite them as guest contributors to Section 3 of the @steem-news column.  There will likely to be no more than 2 guest contributors each week and they will be paid a fee of SBD3 to SBD5 per article.  This will take up approximately 40\% of the budget.
    \item \textbf{Weekly review contest}\\Each weekend we will organise a weekly @steem-news review contest and ask Steemians questions about contents from the three issues during the week.  This can serve as a reminder to them to read @steem-news if they haven't had time to do so during the week.  There will be around 10 questions, and the answers will be submitted via Google form.  The weekly prize pool will be around  SBD10 split between three Steemian who gets the most correct answer, or a lucky draw if there are more than three.  This will take up approximately 35\% of the budget.
\end{enumerate}

\noindent
Members of the project will share 50\% of SBD post payout and the rest will be sold on the internal market for Steem then powered up so as to grow the @steem-news account.  As a result the paper will have more stakes on the platform as well as encourage quality content creators.

\sectionbreak   

\section{Conclusion}

This is not a perfect rendition of all that this paper seeks to achieve and each and every segment is open for correction, amendment and approval from the project members. Since this is a project that will be on the steemit platform for a long time, room for improvement is definitely encouraged. 
\\\\
We believe that this project is something the steem blockchain needs and we believe that we have the wherewithal to propel this project into something that the platform users would enjoy and those who support the project would definitely be glad that they are a part of it. 



\end{document}


